\let\negmedspace\undefined
\let\negthickspace\undefined
%\RequirePackage{amsmath}
\documentclass[journal,12pt,twocolumn]{IEEEtran}
%
% \usepackage{setspace}
% \usepackage{gensymb}
%\doublespacing
%\singlespacing
%\usepackage{silence}
%Disable all warnings issued by latex starting with "You have..."
%\usepackage{graphicx}
\usepackage{amssymb}
%\usepackage{relsize}
\usepackage[cmex10]{amsmath}
%\usepackage{amsthm}
%\interdisplaylinepenalty=2500
%\savesymbol{iint}
%\usepackage{txfonts}
%\restoresymbol{TXF}{iint}
%\usepackage{wasysym}
\usepackage{amsthm}
%\usepackage{iithtlc}
% \usepackage{mathrsfs}
% \usepackage{txfonts}
% \usepackage{stfloats}
% \usepackage{steinmetz}
% \usepackage{bm}
% \usepackage{cite}
% \usepackage{cases}
% \usepackage{subfig}
%\usepackage{xtab}
\usepackage{longtable}
%\usepackage{multirow}
%\usepackage{algorithm}
%\usepackage{algpseudocode}
\usepackage{enumitem}
 \usepackage{mathtools}
% \usepackage{tikz}
% \usepackage{circuitikz}
% \usepackage{verbatim}
%\usepackage{tfrupee}
\usepackage[breaklinks=true]{hyperref}
%\usepackage{stmaryrd}
%\usepackage{tkz-euclide} % loads  TikZ and tkz-base
%\usetkzobj{all}
\usepackage{listings}
    \usepackage{color}                                            %%
    \usepackage{array}                                            %%
    \usepackage{longtable}                                        %%
    \usepackage{calc}                                             %%
    \usepackage{multirow}                                         %%
    \usepackage{hhline}                                           %%
    \usepackage{ifthen}                                           %%
  %optionally (for landscape tables embedded in another document): %%
    \usepackage{lscape}     
% \usepackage{multicol}
% \usepackage{chngcntr}
%\usepackage{enumerate}

%\usepackage{wasysym}
%\newcounter{MYtempeqncnt}
\DeclareMathOperator*{\Res}{Res}
%\renewcommand{\baselinestretch}{2}
\renewcommand\thesection{\arabic{section}}
\renewcommand\thesubsection{\thesection.\arabic{subsection}}
\renewcommand\thesubsubsection{\thesubsection.\arabic{subsubsection}}

\renewcommand\thesectiondis{\arabic{section}}
\renewcommand\thesubsectiondis{\thesectiondis.\arabic{subsection}}
\renewcommand\thesubsubsectiondis{\thesubsectiondis.\arabic{subsubsection}}

% correct bad hyphenation here
\hyphenation{op-tical net-works semi-conduc-tor}
\def\inputGnumericTable{}                                 %%

\lstset{
%language=C,
frame=single, 
breaklines=true,
columns=fullflexible
}
%\lstset{
%language=tex,
%frame=single, 
%breaklines=true
%}

\begin{document}
%


\newtheorem{theorem}{Theorem}[section]
\newtheorem{problem}{Problem}
\newtheorem{proposition}{Proposition}[section]
\newtheorem{lemma}{Lemma}[section]
\newtheorem{corollary}[theorem]{Corollary}
\newtheorem{example}{Example}[section]
\newtheorem{definition}[problem]{Definition}
%\newtheorem{thm}{Theorem}[section] 
%\newtheorem{defn}[thm]{Definition}
%\newtheorem{algorithm}{Algorithm}[section]
%\newtheorem{cor}{Corollary}
\newcommand{\BEQA}{\begin{eqnarray}}
\newcommand{\EEQA}{\end{eqnarray}}
\newcommand{\define}{\stackrel{\triangle}{=}}

\bibliographystyle{IEEEtran}
%\bibliographystyle{ieeetr}


\providecommand{\mbf}{\mathbf}
\providecommand{\pr}[1]{\ensuremath{\Pr\left(#1\right)}}
\providecommand{\qfunc}[1]{\ensuremath{Q\left(#1\right)}}
\providecommand{\sbrak}[1]{\ensuremath{{}\left[#1\right]}}
\providecommand{\lsbrak}[1]{\ensuremath{{}\left[#1\right.}}
\providecommand{\rsbrak}[1]{\ensuremath{{}\left.#1\right]}}
\providecommand{\brak}[1]{\ensuremath{\left(#1\right)}}
\providecommand{\lbrak}[1]{\ensuremath{\left(#1\right.}}
\providecommand{\rbrak}[1]{\ensuremath{\left.#1\right)}}
\providecommand{\cbrak}[1]{\ensuremath{\left\{#1\right\}}}
\providecommand{\lcbrak}[1]{\ensuremath{\left\{#1\right.}}
\providecommand{\rcbrak}[1]{\ensuremath{\left.#1\right\}}}
\theoremstyle{remark}
\newtheorem{rem}{Remark}
\newcommand{\sgn}{\mathop{\mathrm{sgn}}}
\providecommand{\abs}[1]{\left\vert#1\right\vert}
\providecommand{\res}[1]{\Res\displaylimits_{#1}} 
\providecommand{\norm}[1]{\left\lVert#1\right\rVert}
%\providecommand{\norm}[1]{\lVert#1\rVert}
\providecommand{\mtx}[1]{\mathbf{#1}}
\providecommand{\mean}[1]{E\left[ #1 \right]}
\providecommand{\fourier}{\overset{\mathcal{F}}{ \rightleftharpoons}}
%\providecommand{\hilbert}{\overset{\mathcal{H}}{ \rightleftharpoons}}
\providecommand{\system}{\overset{\mathcal{H}}{ \longleftrightarrow}}
	%\newcommand{\solution}[2]{\textbf{Solution:}{#1}}
\newcommand{\solution}{\noindent \textbf{Solution: }}
\newcommand{\cosec}{\,\text{cosec}\,}
\providecommand{\dec}[2]{\ensuremath{\overset{#1}{\underset{#2}{\gtrless}}}}
\newcommand{\myvec}[1]{\ensuremath{\begin{pmatrix} #1\end{pmatrix}}}
\newcommand{\mydet}[1]{\ensuremath{\begin{vmatrix}#1\end{vmatrix}}}
%\numberwithin{equation}{section}
\numberwithin{equation}{subsection}
%\numberwithin{problem}{section}
%\numberwithin{definition}{section}
\makeatletter
\@addtoreset{figure}{problem}
\makeatother

\let\StandardTheFigure\thefigure
\let\vec\mathbf
%\renewcommand{\thefigure}{\theproblem.\arabic{figure}}
\renewcommand{\thefigure}{\theproblem}
%\setlist[enumerate,1]{before=\renewcommand\theequation{\theenumi.\arabic{align}}
%\counterwithin{align}{enumi}


%\renewcommand{\theequation}{\arabic{subsection}.\arabic{align}}

\def\putbox#1#2#3{\makebox[0in][l]{\makebox[#1][l]{}\raisebox{\baselineskip}[0in][0in]{\raisebox{#2}[0in][0in]{#3}}}}
     \def\rightbox#1{\makebox[0in][r]{#1}}
     \def\centbox#1{\makebox[0in]{#1}}
     \def\topbox#1{\raisebox{-\baselineskip}[0in][0in]{#1}}
     \def\midbox#1{\raisebox{-0.5\baselineskip}[0in][0in]{#1}}

\vspace{3cm}

\title{
Matrix Arithmetic
}
\author{ G V V Sharma$^{*}$% <-this % stops a space
	\thanks{*The author is with the Department
		of Electrical Engineering, Indian Institute of Technology, Hyderabad
		502285 India e-mail:  gadepall@iith.ac.in. All content in this manual is released under GNU GPL.  Free and open source.}
	
}	
%\title{
%	\logo{Matrix Analysis through Octave}{\begin{center}\includegraphics[scale=.24]{tlc}\end{center}}{}{HAMDSP}
%}


% paper title
% can use linebreaks  \\ [1em]within to get better formatting as desired
%\title{Matrix Analysis through Octave}
%
%
% author names and IEEE memberships
% note positions of commas and nonbreaking spaces ( ~ ) LaTeX will not break
% a structure at a ~ so this keeps an author's name from being broken across
% two lines.
% use \thanks{} to gain access to the first footnote area
% a separate \thanks must be used for each paragraph as LaTeX2e's \thanks
% was not built to handle multiple paragraphs
%

%\author{<-this % stops a space
%\thanks{}}
%}
% note the % following the last \IEEEmembership and also \thanks - 
% these prevent an unwanted space from occurring between the last author name
% and the end of the author line. i.e., if you had this:
% 
% \author{....lastname \thanks{...} \thanks{...} }
%                     ^------------^------------^----Do not want these spaces!
%
% a space would be appended to the last name and could cause every name on that
% line to be shifted left slightly. This is one of those "LaTeX things". For
% instance, "\textbf{A} \textbf{B}" will typeset as "A B" not "AB". To get
% "AB" then you have to do: "\textbf{A}\textbf{B}"
% \thanks is no different in this regard, so shield the last } of each \thanks
% that ends a line with a % and do not let a space in before the next \thanks.
% Spaces after \IEEEmembership other than the last one are OK (and needed) as
% you are supposed to have spaces between the names. For what it is worth,
% this is a minor point as most people would not even notice if the said evil
% space somehow managed to creep in.

%\WarningFilter{latex}{LaTeX Warning: You have requested, on input line 117, version}


% The paper headers
%\markboth{Journal of \LaTeX\ Class Files,~Vol.~6, No.~1, January~2007}%
%{Shell \MakeLowercase{\textit{et al.}}: Bare Demo of IEEEtran.cls for Journals}
% The only time the second header will appear is for the odd numbered pages
% after the title page when using the twoside option.
% 
% *** Note that you probably will NOT want to include the author's ***
% *** name in the headers of peer review papers.                   ***
% You can use \ifCLASSOPTIONpeerreview for conditional compilation here if
% you desire.




% If you want to put a publisher's ID mark on the page you can do it like
% this:
%\IEEEpubid{0000--0000/00\$00.00~\copyright~2007 IEEE}
% Remember, if you use this you must call \IEEEpubidadjcol in the second
% column for its text to clear the IEEEpubid mark.



% make the title area
\maketitle

\newpage

\tableofcontents

\bigskip

\renewcommand{\thefigure}{\theenumi}
\renewcommand{\thetable}{\theenumi}
%\renewcommand{\theequation}{\theenumi}

%\begin{abstract}
%%\boldmath
%In this letter, an algorithm for evaluating the exact analytical bit error rate  (BER)  for the piecewise linear (PL) combiner for  multiple relays is presented. Previous results were available only for upto three relays. The algorithm is unique in the sense that  the actual mathematical expressions, that are prohibitively large, need not be explicitly obtained. The diversity gain due to multiple relays is shown through plots of the analytical BER, well supported by simulations. 
%
%\end{abstract}
% IEEEtran.cls defaults to using nonbold math in the Abstract.
% This preserves the distinction between vectors and scalars. However,
% if the journal you are submitting to favors bold math in the abstract,
% then you can use LaTeX's standard command \boldmath at the very start
% of the abstract to achieve this. Many IEEE journals frown on math
% in the abstract anyway.

% Note that keywords are not normally used for peerreview papers.
%\begin{IEEEkeywords}
%Cooperative diversity, decode and forward, piecewise linear
%\end{IEEEkeywords}



% For peer review papers, you can put extra information on the cover
% page as needed:
% \ifCLASSOPTIONpeerreview
% \begin{center} \bfseries EDICS Category: 3-BBND \end{center}
% \fi
%
% For peerreview papers, this IEEEtran command inserts a page break and
% creates the second title. It will be ignored for other modes.
%\IEEEpeerreviewmaketitle

\begin{abstract}
This manual has solved examples on arithmetic operations in matrices 
based on exercises from the NCERT textbooks from Class 6-12.  
\end{abstract}
Download python codes using 
\begin{lstlisting}
svn co https://github.com/gadepall/school/trunk/ncert/computation/codes
\end{lstlisting}

%
\section{Examples}
\renewcommand{\theequation}{\theenumi}
%\begin{enumerate}[label=\arabic*.,ref=\theenumi]
\begin{enumerate}[label=\thesection.\arabic*.,ref=\thesection.\theenumi]
\numberwithin{equation}{enumi}

\item Consider the following information regarding the number of men and women workers in the three factories I,II and III

\begin{tabular}{ |c|c|c| } 
\hline
 & Men Workers & Women Workers \\
\hline
\multirow{3}{4em}{I \\II \\III} & 30 & 25\\ 
& 25 & 31 \\ 
&27 & 26 \\ 
\hline
\end{tabular}\\
Represent the above information in the form of a 3 $\times$ 2 matrix. What does the entry
in the third row and second column represent?\\


 \item  If a matrix has 8 elements, what are the possible orders it can have?\\
    \item Construct a 3 $\times$ 2 matrix whose elements are given by $a_{ij}=\frac{1}{2}\abs{i-3j}$\\
    \item \myvec{x+3 &z+4 &2y-7\\-6 &a-1 &0\\b-3 &-21 &0}=\myvec{0 &6 &3y-2\\-6 &-3 &2c+2\\2b+4 &-21 &0}\\
    Find the values of a,b,c,x,y and z.\\
\solution 
\input{./solutions/5/chapters/lines/docq7.tex}
    \item Find the values of a,b,c and d from the following equation:\\
    \myvec{2a+b &a-2b\\5c-d &4c+3d}=\myvec{4 &-3\\11 &24}\\
\solution 
\input{./solutions/matrix/examples/6/chapters/solution.tex}
    \item Given A=\myvec{\sqrt{3} &1 &-1\\2 &3 &0} and B=\myvec{2 &\sqrt{5} &1\\-2 &3 &\frac{1}{2}}, find A+B.\\\solution 
\input{./solutions/7/chapters/line/matrix/solution.tex}
\item In the matrix A=\myvec{2 &5 &19 &-7\\ 35 &-2 &\frac{5}{2} &12 \\ \sqrt{3} &1 &-5 &17}, write
\begin{enumerate}
\item The order of the matrix
\item The number of elements
\item Write the elements $a_{31},a_{21},a_{33},a_{24},a_{23}.$
\end{enumerate}
\solution 
\input{./solutions/1/chapters/line/matrix/solution.tex}

\item If a matrix has 24 elements,what are the possible orders it can have? What,if it has 13 elements?\\
\solution 
\input{./solutions/2/chapters/line_ex/matrix/solution.tex}
\item If a matrix has 18 elements,what are the possible orders it can have? What,if it has 5 elements?\\
\\
\solution 
\input{./solutions/3/chapters/line/matrix/solution.tex}
%
\item Construct a $2 \times 2$ matrix,A=[$a_{ij}$],whose elements are given by:\\
(i) $a_{ij}$=$\frac{(i+j)^2}{2}$\ (ii) $a_{ij}$=$\frac{i}{j}$\ (iii) $a_{ij}$=$\frac{(i+2j)^2}{2}$\\
\solution 
\input{./solutions/4/chapters/line/matrix/solution.tex}
\item Construct a $3\times 4$ matrix,whose elements are given by:\\
(i) $a_{ij}$=$\frac{1}{2}\abs{-3i+j}$ (ii) $a_{ij}$=2i-j\\
\solution 
\input{./solutions/5/chapters/lines/docq13.tex}
\item Find the values of x,y and z from the following equations:\\
(i) \myvec{4 &3\\x &5} = \myvec{y &z\\1 &5} (ii) \myvec{x+y &2\\5+z &xy} = \myvec{6 &2\\5 &8} (iii) \myvec{x+y+z\\x+y\\y+z}=\myvec{9\\5\\7}\\
\\
\solution 
\input{./solutions/6/chapters/line/matrix/solution.tex}
\item Compute the indicated product.
\item \myvec{1\\2\\3}\myvec{2 &3 &4} 
\\
\solution 
\input{./solutions/matrix/12_2/solution.tex}
\item \myvec{1 &-2\\2 &3}\myvec{1 &2 &3\\2 &3 &1} \\
\end{enumerate}
\item Simplify $\cos\theta$$\myvec{\cos\theta &\sin\theta\\ -\sin\theta &\cos\theta}$+$\sin\theta$$\myvec{\sin\theta &-\cos\theta\\ \cos\theta &\sin\theta}$\\
\solution 
\input{./solutions/matrix/16/solution.tex}
\item If F(x)=$\myvec{\cos x &-\sin x &0\\ \sin x &\cos x &0\\0 &0 &1}$\\,show that F(x)F(y)=F(x+y)\\
\solution 
\input{./solutions/matrix/23/solution.tex}

\item For the matrices A and B,verify that $(AB)^{'}$=$B^{'}A^{'}$,where\\
(i)A=\myvec{1\\-4\\3},B=\myvec{-1 &2 &1} (ii)A=\myvec{0\\1\\2},B=\myvec{1 &5 &7}

  \item For what values of x: \\\myvec{1 &2 &1}\myvec{1 &2 &0\\2 &0 &1\\1 &0 &2}\myvec{0 \\2 \\x}=0?\\
\solution
\input{./solutions/matrix/69/chapters/solution.tex}
  \item A manufactrer produces three products x,y,z which he sells in two markets. Annual sales are indicated below:\\
 
  \begin{tabular}{cccc}
  \hline
  Market & Products\\
  \hline
  I &10,000 &2,000 &18,000\\
  \hline
  II &6,000 &20,000 &8,000\\
  \hline
  \end{tabular}\\
  (a) If unit sale prices of x,y and z are \rupee{2.50},\rupee{1.50} and \rupee{1.00} respectively,find the total revenue in each market with the help of matrix algebra.\\
  (b) If the unit cost of the above three commodities are \rupee{2.00},\rupee{1.00} and 50 paise respectively.Find the gross profit.\\
  \solution
  \input{solutions/su2021/45/assignment10.tex}
\item     Two farmers Ramkishan and Gurcharan Singh cultivate only three varieties of rice namely Basmati, Permal and Naura. The sale (in Rupees) of these varieties of rice by both the farmers in the month of September and October are given by the following matrices $\vec{A}$ and $\vec{B}$ .

\begin{center}
September Sales(in Rupees)
\end{center}
\begin{align}
    \vec{A}=
    \begin{blockarray}{cccc}
    \text{Basmati} & \text{Permal} & \text{Naura} \\
    \begin{block}{(ccc)(c)}
    10000 & 20000 & 30000 & \text{Ramkishan}\\
    50000 & 30000 & 10000 & \text{Gurucharan} \\
    \end{block}
    \end{blockarray}
\end{align}

\begin{center}
October Sales(in Rupees)
\end{center}
\begin{align}
    \vec{B} =
    \begin{blockarray}{cccc}
    \text{Basmati} & \text{Permal} & \text{Naura} \\
    \begin{block}{(ccc)(c)}
    5000 & 10000 & 6000 & \text{Ramkishan}\\
    20000 & 10000 & 10000 & \text{Gurucharan} \\
    \end{block}
    \end{blockarray}
\end{align}

\begin{enumerate}
    \item Find the combined sales in September and October for each farmer in each variety.
    \item Find the decrease in sales from September to October.
    \item If both farmers receive 2\% profit on gross sales, compute the profit for each farmer and for each variety sold in October.
\end{enumerate}
%
\solution
\input{solutions/su2021/56/Assignment10.tex}
\item In a legislative assembly election, a political
group hired a public relations firm to promote
its candidate in three ways: telephone, house
calls, and letters. The cost per contact (in paise)
is given in matrix $\vec{A}$ as
\begin{center}
Cost per Contact(in Paise)
\end{center}
\begin{align}
    \vec{A}=
    \begin{blockarray}{cc}
    \text{cost}\\
    \begin{block}{(c)(c)}
    40 &\text{Telephone}\\
    100&\text{Housecall} \\
    50&\text{Letter}\\
    \end{block}
    \end{blockarray}
\end{align}
The number of contacts of each type made in
two cities X and Y is given by matrix $\vec{B}$
\begin{align}
    \vec{B} =
    \begin{blockarray}{cccc}
    \text{Telephone} & \text{Housecall} & \text{Letter} \\
    \begin{block}{(ccc)(c)}
    1000 & 500 & 5000 & \text{X}\\
    3000 & 1000 & 10000 & \text{Y} \\
    \end{block}
    \end{blockarray}
\end{align}
Find the total
amount spent by the group in the two cities
X and Y
%
\solution
\input{solutions/su2021/64/Assignment-8.tex}
\item Express the matrix B=$\myvec{2 &-2 &-4\\-1 &3 &4\\1 &-2 &-3}$ as the sum of a symmetric and a skew symmetric matrix.\\
\solution
  \input{solutions/su2021/67/main.tex}
\end{enumerate}
\end{document}

