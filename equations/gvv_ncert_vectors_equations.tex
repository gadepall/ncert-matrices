\let\negmedspace\undefined
\let\negthickspace\undefined
%\RequirePackage{amsmath}
\documentclass[journal,12pt,twocolumn]{IEEEtran}
%
% \usepackage{setspace}
% \usepackage{gensymb}
%\doublespacing
%\singlespacing
%\usepackage{silence}
%Disable all warnings issued by latex starting with "You have..."
%\usepackage{graphicx}
%\usepackage{amssymb}
%\usepackage{relsize}
\usepackage[cmex10]{amsmath}
%\usepackage{amsthm}
%\interdisplaylinepenalty=2500
%\savesymbol{iint}
%\usepackage{txfonts}
%\restoresymbol{TXF}{iint}
%\usepackage{wasysym}
\usepackage{amsthm}
%\usepackage{iithtlc}
% \usepackage{mathrsfs}
% \usepackage{txfonts}
% \usepackage{stfloats}
% \usepackage{steinmetz}
% \usepackage{bm}
% \usepackage{cite}
% \usepackage{cases}
% \usepackage{subfig}
%\usepackage{xtab}
\usepackage{longtable}
%\usepackage{multirow}
%\usepackage{algorithm}
%\usepackage{algpseudocode}
\usepackage{enumitem}
 \usepackage{mathtools}
% \usepackage{tikz}
% \usepackage{circuitikz}
% \usepackage{verbatim}
%\usepackage{tfrupee}
\usepackage[breaklinks=true]{hyperref}
%\usepackage{stmaryrd}
%\usepackage{tkz-euclide} % loads  TikZ and tkz-base
%\usetkzobj{all}
\usepackage{listings}
    \usepackage{color}                                            %%
    \usepackage{array}                                            %%
    \usepackage{longtable}                                        %%
    \usepackage{calc}                                             %%
    \usepackage{multirow}                                         %%
    \usepackage{hhline}                                           %%
    \usepackage{ifthen}                                           %%
  %optionally (for landscape tables embedded in another document): %%
    \usepackage{lscape}     
% \usepackage{multicol}
% \usepackage{chngcntr}
%\usepackage{enumerate}

%\usepackage{wasysym}
%\newcounter{MYtempeqncnt}
\DeclareMathOperator*{\Res}{Res}
%\renewcommand{\baselinestretch}{2}
\renewcommand\thesection{\arabic{section}}
\renewcommand\thesubsection{\thesection.\arabic{subsection}}
\renewcommand\thesubsubsection{\thesubsection.\arabic{subsubsection}}

\renewcommand\thesectiondis{\arabic{section}}
\renewcommand\thesubsectiondis{\thesectiondis.\arabic{subsection}}
\renewcommand\thesubsubsectiondis{\thesubsectiondis.\arabic{subsubsection}}

% correct bad hyphenation here
\hyphenation{op-tical net-works semi-conduc-tor}
\def\inputGnumericTable{}                                 %%

\lstset{
%language=C,
frame=single, 
breaklines=true,
columns=fullflexible
}
%\lstset{
%language=tex,
%frame=single, 
%breaklines=true
%}

\begin{document}
%


\newtheorem{theorem}{Theorem}[section]
\newtheorem{problem}{Problem}
\newtheorem{proposition}{Proposition}[section]
\newtheorem{lemma}{Lemma}[section]
\newtheorem{corollary}[theorem]{Corollary}
\newtheorem{example}{Example}[section]
\newtheorem{definition}[problem]{Definition}
%\newtheorem{thm}{Theorem}[section] 
%\newtheorem{defn}[thm]{Definition}
%\newtheorem{algorithm}{Algorithm}[section]
%\newtheorem{cor}{Corollary}
\newcommand{\BEQA}{\begin{eqnarray}}
\newcommand{\EEQA}{\end{eqnarray}}
\newcommand{\define}{\stackrel{\triangle}{=}}

\bibliographystyle{IEEEtran}
%\bibliographystyle{ieeetr}


\providecommand{\mbf}{\mathbf}
\providecommand{\pr}[1]{\ensuremath{\Pr\left(#1\right)}}
\providecommand{\qfunc}[1]{\ensuremath{Q\left(#1\right)}}
\providecommand{\sbrak}[1]{\ensuremath{{}\left[#1\right]}}
\providecommand{\lsbrak}[1]{\ensuremath{{}\left[#1\right.}}
\providecommand{\rsbrak}[1]{\ensuremath{{}\left.#1\right]}}
\providecommand{\brak}[1]{\ensuremath{\left(#1\right)}}
\providecommand{\lbrak}[1]{\ensuremath{\left(#1\right.}}
\providecommand{\rbrak}[1]{\ensuremath{\left.#1\right)}}
\providecommand{\cbrak}[1]{\ensuremath{\left\{#1\right\}}}
\providecommand{\lcbrak}[1]{\ensuremath{\left\{#1\right.}}
\providecommand{\rcbrak}[1]{\ensuremath{\left.#1\right\}}}
\theoremstyle{remark}
\newtheorem{rem}{Remark}
\newcommand{\sgn}{\mathop{\mathrm{sgn}}}
\providecommand{\abs}[1]{\left\vert#1\right\vert}
\providecommand{\res}[1]{\Res\displaylimits_{#1}} 
\providecommand{\norm}[1]{\left\lVert#1\right\rVert}
%\providecommand{\norm}[1]{\lVert#1\rVert}
\providecommand{\mtx}[1]{\mathbf{#1}}
\providecommand{\mean}[1]{E\left[ #1 \right]}
\providecommand{\fourier}{\overset{\mathcal{F}}{ \rightleftharpoons}}
%\providecommand{\hilbert}{\overset{\mathcal{H}}{ \rightleftharpoons}}
\providecommand{\system}{\overset{\mathcal{H}}{ \longleftrightarrow}}
	%\newcommand{\solution}[2]{\textbf{Solution:}{#1}}
\newcommand{\solution}{\noindent \textbf{Solution: }}
\newcommand{\cosec}{\,\text{cosec}\,}
\providecommand{\dec}[2]{\ensuremath{\overset{#1}{\underset{#2}{\gtrless}}}}
\newcommand{\myvec}[1]{\ensuremath{\begin{pmatrix}#1\end{pmatrix}}}
\newcommand{\mydet}[1]{\ensuremath{\begin{vmatrix}#1\end{vmatrix}}}
%\numberwithin{equation}{section}
\numberwithin{equation}{subsection}
%\numberwithin{problem}{section}
%\numberwithin{definition}{section}
\makeatletter
\@addtoreset{figure}{problem}
\makeatother

\let\StandardTheFigure\thefigure
\let\vec\mathbf
%\renewcommand{\thefigure}{\theproblem.\arabic{figure}}
\renewcommand{\thefigure}{\theproblem}
%\setlist[enumerate,1]{before=\renewcommand\theequation{\theenumi.\arabic{equation}}
%\counterwithin{equation}{enumi}


%\renewcommand{\theequation}{\arabic{subsection}.\arabic{equation}}

\def\putbox#1#2#3{\makebox[0in][l]{\makebox[#1][l]{}\raisebox{\baselineskip}[0in][0in]{\raisebox{#2}[0in][0in]{#3}}}}
     \def\rightbox#1{\makebox[0in][r]{#1}}
     \def\centbox#1{\makebox[0in]{#1}}
     \def\topbox#1{\raisebox{-\baselineskip}[0in][0in]{#1}}
     \def\midbox#1{\raisebox{-0.5\baselineskip}[0in][0in]{#1}}

\vspace{3cm}

\title{
	%\logo{
%Computational Approach to School Geometry
Points and Vectors
%	}
}
\author{ G V V Sharma$^{*}$% <-this % stops a space
	\thanks{*The author is with the Department
		of Electrical Engineering, Indian Institute of Technology, Hyderabad
		502285 India e-mail:  gadepall@iith.ac.in. All content in this manual is released under GNU GPL.  Free and open source.}
	
}	
%\title{
%	\logo{Matrix Analysis through Octave}{\begin{center}\includegraphics[scale=.24]{tlc}\end{center}}{}{HAMDSP}
%}


% paper title
% can use linebreaks \\ within to get better formatting as desired
%\title{Matrix Analysis through Octave}
%
%
% author names and IEEE memberships
% note positions of commas and nonbreaking spaces ( ~ ) LaTeX will not break
% a structure at a ~ so this keeps an author's name from being broken across
% two lines.
% use \thanks{} to gain access to the first footnote area
% a separate \thanks must be used for each paragraph as LaTeX2e's \thanks
% was not built to handle multiple paragraphs
%

%\author{<-this % stops a space
%\thanks{}}
%}
% note the % following the last \IEEEmembership and also \thanks - 
% these prevent an unwanted space from occurring between the last author name
% and the end of the author line. i.e., if you had this:
% 
% \author{....lastname \thanks{...} \thanks{...} }
%                     ^------------^------------^----Do not want these spaces!
%
% a space would be appended to the last name and could cause every name on that
% line to be shifted left slightly. This is one of those "LaTeX things". For
% instance, "\textbf{A} \textbf{B}" will typeset as "A B" not "AB". To get
% "AB" then you have to do: "\textbf{A}\textbf{B}"
% \thanks is no different in this regard, so shield the last } of each \thanks
% that ends a line with a % and do not let a space in before the next \thanks.
% Spaces after \IEEEmembership other than the last one are OK (and needed) as
% you are supposed to have spaces between the names. For what it is worth,
% this is a minor point as most people would not even notice if the said evil
% space somehow managed to creep in.

%\WarningFilter{latex}{LaTeX Warning: You have requested, on input line 117, version}


% The paper headers
%\markboth{Journal of \LaTeX\ Class Files,~Vol.~6, No.~1, January~2007}%
%{Shell \MakeLowercase{\textit{et al.}}: Bare Demo of IEEEtran.cls for Journals}
% The only time the second header will appear is for the odd numbered pages
% after the title page when using the twoside option.
% 
% *** Note that you probably will NOT want to include the author's ***
% *** name in the headers of peer review papers.                   ***
% You can use \ifCLASSOPTIONpeerreview for conditional compilation here if
% you desire.




% If you want to put a publisher's ID mark on the page you can do it like
% this:
%\IEEEpubid{0000--0000/00\$00.00~\copyright~2007 IEEE}
% Remember, if you use this you must call \IEEEpubidadjcol in the second
% column for its text to clear the IEEEpubid mark.



% make the title area
\maketitle

\newpage

\tableofcontents

\bigskip

\renewcommand{\thefigure}{\theenumi}
\renewcommand{\thetable}{\theenumi}
%\renewcommand{\theequation}{\theenumi}

%\begin{abstract}
%%\boldmath
%In this letter, an algorithm for evaluating the exact analytical bit error rate  (BER)  for the piecewise linear (PL) combiner for  multiple relays is presented. Previous results were available only for upto three relays. The algorithm is unique in the sense that  the actual mathematical expressions, that are prohibitively large, need not be explicitly obtained. The diversity gain due to multiple relays is shown through plots of the analytical BER, well supported by simulations. 
%
%\end{abstract}
% IEEEtran.cls defaults to using nonbold math in the Abstract.
% This preserves the distinction between vectors and scalars. However,
% if the journal you are submitting to favors bold math in the abstract,
% then you can use LaTeX's standard command \boldmath at the very start
% of the abstract to achieve this. Many IEEE journals frown on math
% in the abstract anyway.

% Note that keywords are not normally used for peerreview papers.
%\begin{IEEEkeywords}
%Cooperative diversity, decode and forward, piecewise linear
%\end{IEEEkeywords}



% For peer review papers, you can put extra information on the cover
% page as needed:
% \ifCLASSOPTIONpeerreview
% \begin{center} \bfseries EDICS Category: 3-BBND \end{center}
% \fi
%
% For peerreview papers, this IEEEtran command inserts a page break and
% creates the second title. It will be ignored for other modes.
%\IEEEpeerreviewmaketitle

\begin{abstract}
This book provides a computational approach to school geometry based on the NCERT textbooks from Class 6-12.  Links to sample Python codes are available in the text.  
\end{abstract}

\section{Linear Equations}
\renewcommand{\theequation}{\theenumi}
%\begin{enumerate}[label=\arabic*.,ref=\theenumi]
\begin{enumerate}[label=\thesection.\arabic*.,ref=\thesection.\theenumi]
\numberwithin{equation}{enumi}
\item Solve the following
%
\begin{enumerate}[itemsep=2pt]
\begin{multicols}{2}
\item
\begin{align}
\begin{split}
\myvec{1 & 1 }\vec{x}&=5
\\
\myvec{2 & -3 }\vec{x}&=4
\end{split}
\end{align}
\item
\begin{align}
\begin{split}
\myvec{3 & 4 }\vec{x}&=10
\\
\myvec{2 & -2 }\vec{x}&=2
\end{split}
\end{align}
\item
\begin{align}
\begin{split}
\myvec{3 & -5 }\vec{x}&=4
\\
\myvec{9 & -2 }\vec{x}&=7
\end{split}
\end{align}
\begin{align}
\begin{split}
\myvec{\frac{1}{2} & \frac{2}{3} }\vec{x}&=-1
\\
\myvec{{1} & -\frac{1}{3} }\vec{x}&=3
\end{split}
\end{align}
\end{multicols}
\end{enumerate}
%
\solution 
\input{./solutions/line_plane/18/chapters/solution.tex}

\item Solve the following pair of linear equations
\begin{align}
\begin{split}
\myvec{8 & 5 }\vec{x}&=9
\\
\myvec{3 & 2 }\vec{x}&=4
\end{split}
\end{align}
\solution 
\input{./solutions/line_plane/22/solution.tex}
%
\item Solve the following pair of linear equations
\begin{align}
\begin{split}
\myvec{158 & -378 }\vec{x}&=-74
\\
\myvec{-378 & 152 }\vec{x}&=-604
\end{split}
\end{align}
\solution 
\input{./solutions/line_plane/23/solution.tex}

\item Solve the following pair of equations
%
\begin{align}
\label{eq:line_check_sol_unique}
\myvec{7 & -15}\vec{x}  &= 2 
\\
\myvec{ 1 & 2}\vec{x} &= 3
\end{align}
%
\\
\solution The above equations can be expressed as the matrix equation
\begin{align}
\myvec{7 & -15\\1 & 2} \vec{x} = \myvec{2\\ 3}
\end{align}
%
The augmented matrix for the above equation is row reduced as follows
\begin{align}
\myvec{7 & -15 & 2\\1 & 2 & 3}
\xleftrightarrow {R_2\leftarrow 7R_2-R_1}\myvec{7 & -15 & 2\\0 & 29 & 19}
\\
\xleftrightarrow {R_1\leftarrow \frac{15R_2 29+ 29R_1}{29}}\myvec{7 & 0 & 2\\0 & 29 & 19} 
\\
\implies \vec{x} = \myvec{\frac{2}{7}\\\frac{19}{29}}
\end{align}
%
%

The python code in Problem \ref{prob:line_mat_eq}
%
%
can be used to plot Fig. \ref{fig:line_check_sol_unique}, which shows that the lines are the same.
%
\begin{figure}[!ht]
\includegraphics[width=\columnwidth]{./line/figs/line_check_unique.eps}
\caption{}
\label{fig:line_check_sol_unique}
\end{figure}
%
\item Find all possibe solutions of
\begin{align}
\label{eq:line_check_sol_nosol}
\begin{split}
\myvec{2 & 3 }\vec{x}&=8
\\
\myvec{4 & 6 }\vec{x}&=7
\end{split}
\end{align}
%
\\
\solution The above equations can be expressed as the matrix equation
\begin{align}
\myvec{2 & -3\\4 & 6} \vec{x} = \myvec{8\\ 7}
\end{align}
%
The augmented matrix for the above equation is row reduced as follows
\begin{align}
\myvec{2 & 3 & 8\\4 &  6 & 7} 
\xleftrightarrow {R_2\leftarrow R_2-2R_1}\myvec{2 & -3 & 8\\0 &  0 & -9} 
\\
\implies rank \myvec{2 & -3\\4 & 6} \ne \myvec{2 & 3 & 8\\4 &  6 & 7}. 
\end{align}
%
Hence, \eqref{eq:line_check_sol_nosol} has no solution.
The python code in Problem \ref{prob:line_mat_eq}
%
%
can be used to plot Fig. \ref{fig:line_check_nosol}, which shows that the lines are parallel.
%
\begin{figure}[!ht]
\includegraphics[width=\columnwidth]{./line/figs/line_check_nosol.eps}
\caption{}
\label{fig:line_check_nosol}
\end{figure}

\item Which of the following pairs of linear equations are consistent/inconsistent? If consistent, obtain the solution:
%
\begin{enumerate}[itemsep=2pt]
%\begin{multicols}{2}
\item
\begin{align}
\begin{split}
\myvec{1 & 1 }\vec{x}&=5
\\
\myvec{2 & 2 }\vec{x}&=10
\end{split}
\label{linform/2/8/1.0.1}
\end{align}
\item
\begin{align}
\begin{split}
\myvec{1 & -1 }\vec{x}&=8
\\
\myvec{3 & -3 }\vec{x}&=16
\end{split}
\label{linform/2/8/1.0.2}
\end{align}
\item
\begin{align}
\begin{split}
\myvec{2 & 1 }\vec{x}&=6
\\
\myvec{4 & -2 }\vec{x}&=4
\end{split}
\label{linform/2/8/2/1.0.1}
\end{align}
\item
\begin{align}
\begin{split}
\myvec{2 & -2 }\vec{x}&=2
\\
\myvec{4 & -4 }\vec{x}&=5
\end{split}
\label{linform/2/8/2/1.0.2}
\end{align}
%\end{multicols}
\end{enumerate}
%
\solution
\input{solutions/su2021/2/8/ASSIGNMENT2/main.tex}
\input{solutions/su2021/2/8/2/main.tex}
\item Find the intersection of the following lines
%
\begin{enumerate}[itemsep=2pt]
%\begin{multicols}{2}
\item
\begin{align}
\begin{split}
\myvec{1 & 1 }\vec{x}&=14
\\
\myvec{1 & -1 }\vec{x}&=4
\end{split}
\label{linform/10/ab/1.0.1}
\end{align}
%
%
\item
\begin{align}
\begin{split}
\myvec{1 & -1 }\vec{x}&=3
\\
\myvec{\frac{1}{3} & \frac{1}{2} }\vec{x}&=6
\end{split}
\label{linform/10/ab/1.0.2}
\end{align}
\item
\begin{align}
\begin{split}
\myvec{3 & -1 }\vec{x}&=3
\\
\myvec{9 & -3 }\vec{x}&=9
\end{split}
\label{linform/10/1.0.1}
\end{align}
\item
\begin{align}
\begin{split}
\myvec{0.2 & 0.3 }\vec{x}&=1.3
\\
\myvec{0.4 & 0.5 }\vec{x}&=2.3
\end{split}
\label{linform/10/1.0.2}
\end{align}
\item
\begin{align}
\begin{split}
\myvec{\sqrt{2} & \sqrt{3} }\vec{x}&=0
\\
\myvec{\sqrt{3} & \sqrt{8} }\vec{x}&=0
\end{split}
\end{align}
\item
\begin{align}
\begin{split}
\myvec{\frac{3}{2} & -\frac{5}{3} }\vec{x}&=-2
\\
\myvec{\frac{1}{3} & \frac{1}{2} }\vec{x}&=\frac{13}{6}
\end{split}
\end{align}
%\end{multicols}
\end{enumerate}
%
\solution

\input{solutions/su2021/2/10/ab/main.tex}
%
\input{solutions/su2021/2/10/Assignment 2.tex}
\input{solutions/su2021/2/10/ef/main.tex}
\item Draw the graphs of the equations 
\begin{align}
\label{eq:1.2.1_p1}
\myvec{1 & -1}\vec{x} + 1 &= 0 
\\
\myvec{ 3 & 2}\vec{x} - 12 &= 0
\label{eq:1.2.1_p2}
\end{align}
%
  Determine the coordinates of the vertices of the triangle formed by these lines and the x-axis, and shade the triangular region.
\\
\solution
%\input{./solutions/1/chapters/triangle/solution.tex}
%
\item In a $\triangle ABC, \angle C = 3 \angle B = 2 (\angle A + \angle B)$. Find the three angles. 
\\
\solution
%\input{./solutions/2/chapters/triangle_ex/solution.tex}
\item Draw the graphs of the equations $5x – y = 5$ and $3x – y = 3$. Determine the co-ordinates of the vertices of the triangle formed by these lines and the y axis.
\\
\solution
%\input{./solutions/3/chapters/triangle/solution.tex}

\item $ABCD$ is a cyclic quadrilateral with 
\begin{align}
\angle A &= 4y+20
\\
\angle B &= 3y-5
\\
\angle C &= -4x
\\
\angle D &= -7x+5
\end{align}
%
Find its angles.
\\
\solution
%\input{./solutions/2/chapters/quadilateral_ex/solution.tex}
\end{enumerate}
\section{Rank}
\renewcommand{\theequation}{\theenumi}
%\begin{enumerate}[label=\arabic*.,ref=\theenumi]
\begin{enumerate}[label=\thesection.\arabic*.,ref=\thesection.\theenumi]
\numberwithin{equation}{enumi}
\item Find $m$ if 
\begin{align}
\begin{split}
\myvec{2 & 3 }\vec{x}&=11
\\
\myvec{2 & -4 }\vec{x}&=-24
\\
\myvec{m & -1 }\vec{x}&=-3
\\
\end{split}
\end{align}
%
\solution 
\input{./solutions/line_plane/17/solution.tex}
\item For which values of $a$ and $b$ does the following pair of linear equations have an infinite number of solutions?
\begin{align}
\begin{split}
\myvec{2 & 3 }\vec{x}&=7
\\
\myvec{a-b & a+b }\vec{x}&=3a+b-2
\end{split}
\end{align}
\solution 
\input{./solutions/line_plane/20/solution.tex}
%
\item For which value of $k$ will the following pair of linear equations have no solution?
\begin{align}
\begin{split}
\myvec{3 & 1 }\vec{x}&=1
\\
\myvec{2k-1 & k-1 }\vec{x}&=2k+1
\end{split}
\end{align}
\\
\solution
\input{./solutions/line_plane/21/solution.tex}
\item If the lines 
\begin{align}
\myvec{2 & 1}\vec{x}  = 3
\\
\myvec{5 & k}\vec{x}  = 3
\\
\myvec{3 & -1}\vec{x}  = 2
\end{align}
%
are concurrent, find the value of $k$.
%
\\
\solution If the lines are concurrent, the {\em augmented}  matrix should have a 0 row upon row reduction.  Hence, 
%
\begin{align}
\myvec{
2 & 1 & 3
\\
5 & k & 3
\\
3 & -1 & 2
}
\xleftrightarrow{R_2\leftrightarrow R_3}
\myvec{
2 & 1 & 3
\\
3 & -1 & 2
\\
5 & k & 3
}
\\
\xleftrightarrow [R_3\leftarrow 2R_3-5R_1]{R_2\leftrightarrow 2R_2-3R_1}
\myvec{
2 & 1 & 3
\\
0 & -5 & -5
\\
0 & 2k-5 & -9
}
\\
\xleftrightarrow []{R_2\leftarrow -\frac{R_2}{5}}
\myvec{
2 & 1 & 3
\\
0 & 1 & 1
\\
0 & 2k-5 & -9
}
\\
\xleftrightarrow []{R_3\leftarrow R_3-\brak{2k-5}R_2}
\myvec{
2 & 1 & 3
\\
0 & 1 & 1
\\
0 & 0 & -2k-4
}
\\
\implies k = -2
\end{align}
\item For which values of p does the pair of equations given below has unique solution?
\begin{align}
\label{eq:line_det_unique}
\begin{split}
\myvec{4 & p }\vec{x}&=-8
\\
\myvec{2 & 2 }\vec{x}&=-2
\end{split}
\end{align}
%
\solution \eqref{eq:line_det_unique} has a unique solution 
\begin{align}
\iff \mydet{4 & p \\ 2 & 2} \ne 0
\\
\text{or, } p \ne 4
\end{align}
%
\item For what values of $k$ will the following pair of linear equations have infinitely many solutions?
%
\begin{align}
\label{eq:line_det_inf}
\begin{split}
\myvec{k & 3 }\vec{x}&=k-3
\\
\myvec{12 & k }\vec{x}&=k
\end{split}
\end{align}
%
\solution The first condition for \eqref{eq:line_det_inf} to have infinite solutions is 
%
\begin{align}
\label{eq:line_det_inf_cond}
\mydet{k & 3 \\12 & k  } &= 0
\\
\implies k^2 = 36, \text{or, } k = \pm 6
\end{align}
%
For $k = 6$, 
%
the augmented matrix for the above equation is row reduced as follows
\begin{align}
\myvec{6 & 3 & 3\\12 &  6 & 6} 
\xleftrightarrow {R_2\leftarrow R_2-2R_1}\myvec{6 & 3 & 3\\0 &  0 & 0} 
\end{align}
%
indicating that \eqref{eq:line_det_inf} has infinite number of solutions. For $k = -6$, the augmented matrix is 
\begin{align}
\myvec{6 & 3 & -9\\12 &  6 & -6} 
\xleftrightarrow {R_2\leftarrow R_2-2R_1}\myvec{6 & 3 & -9\\0 &  0 & 12} 
\end{align}
indicating that \eqref{eq:line_det_inf} has no solution
%
Thus, \eqref{eq:line_det_inf_cond} is a necessary condition but not sufficient.
\item Check whether the pair of equations 
\begin{align}
\label{eq:line_mat_eq_2}
\begin{split}
\myvec{1 & 3}\vec{x}  &= 6 \text{ and}
\\
\myvec{ 2 & -3}\vec{x} &= 12 
\end{split}
\end{align}
%
is consistent. 
\\
\solution The above equations can be expressed as the matrix equation
\begin{align}
\myvec{1 & 3\\2 & -3} \vec{x} = \myvec{6\\12}
\end{align}
%
The augmented matrix for the above equation is row reduced as follows
\begin{align}
\myvec{1 & 3 & 6\\2 & -3 & 12} 
\xleftrightarrow {R_2\leftarrow \frac{R_2-2R_1}{-9}}
%\myvec{1 & 3 & 6\\0 & -9 & 0} 
\\
\myvec{1 & 3 & 6\\0 & 1 & 0} 
\xleftrightarrow {R_1\leftarrow R_1 - 3R_2}\myvec{1 & 0 & 6\\0 & 1 & 0} 
\\
\implies \vec{x} = \myvec{6\\0}
\end{align}
%
which is the solution of \ref{eq:line_mat_eq}.
The python code in Problem \ref{prob:line_mat_eq}
%
%
can be used to plot Fig. \ref{fig:line_check_sol2}, which shows that the lines intersect.
%
\begin{figure}[!ht]
\includegraphics[width=\columnwidth]{./line/figs/line_check_sol2.eps}
\caption{}
\label{fig:line_check_sol2}
\end{figure}
%
%
\item Find whether the following pair of equations has no solution, unique solution or infinitely many solutions: 
%
\begin{align}
\label{eq:line_check_sol3}
\begin{split}
\myvec{5 & -8}\vec{x}  &= -1 \text{ and}
\\
\myvec{ 3 & -\frac{24}{5}}\vec{x} &= -\frac{3}{5}
\end{split}
\end{align}
%
\\
\solution The above equations can be expressed as the matrix equation
\begin{align}
\myvec{5 & -8\\3 & -\frac{24}{5}} \vec{x} = -\myvec{1\\ \frac{3}{5}}
\end{align}
%
The augmented matrix for the above equation is row reduced as follows
\begin{align}
\myvec{5 & -8 & -1 \\3 & -\frac{24}{5} & -\frac{3}{5}} 
\xleftrightarrow {R_2\leftarrow 5R_2}\myvec{5 & -8 & 1\\15 & -24 & -3} 
\\
%\myvec{5 & -8 & 1\\15 & -24 & -3} 
\xleftrightarrow {R_2\leftarrow R_2 - 3R_1}\myvec{5 & -8 & 1\\0 & 0 & 0} 
\end{align}
%
%
\begin{align}
\because rank \myvec{5 & -8\\3 & -\frac{24}{5}} &= rank \myvec{5 & -8 & 1 \\3 & -\frac{24}{5} & -\frac{3}{5}} 
\\
= 1 < dim \myvec{5 & -8\\3 & -\frac{24}{5}} =2,
\end{align}
%
\eqref{eq:line_check_sol3} has infinitely many solutions.
%
The python code in Problem \ref{prob:line_mat_eq}
%
%
can be used to plot Fig. \ref{fig:line_check_sol3}, which shows that the lines are the same.
%
\begin{figure}[!ht]
\includegraphics[width=\columnwidth]{./line/figs/line_check_sol3.eps}
\caption{}
\label{fig:line_check_sol3}
\end{figure}
%
\item Two rails are represented by the equations 
\label{prob:line_mat_eq}
\begin{align}
\label{eq:line_mat_eq}
\begin{split}
\myvec{1 & 2}\vec{x}  &= 4 \text{ and}
\\
\myvec{ 2 & 4}\vec{x} &=  12 . 
\end{split}
\end{align}
%
Will the rails cross each other?
%
\\
\solution The above equations can be expressed as the matrix equation
\begin{align}
\myvec{1 & 2\\2 & 4} \vec{x} = \myvec{4\\12}
\end{align}
%
The augmented matrix for the above equation is row reduced as follows
\begin{align}
\myvec{1 & 2 & 4\\2 & 4 & 12} 
\xleftrightarrow {R_2\leftarrow \frac{R_2}{2}}\myvec{1 & 2 & 4\\1 & 2 & 6} 
\\
%\myvec{1 & 2 & 4\\2 & 4 & 12} 
\xleftrightarrow {R_2\leftarrow R_2 - R_1}\myvec{1 & 2 & 4\\0 & 0 & 2} 
\label{eq:line_aug}
\end{align}
%
$\because$ row reduction of the $2\times 3$ matrix
%
\begin{align}
\myvec{1 & 2 & 4\\2 & 4 & 12} 
\end{align}
%
results in a matrix with 2 nonzero rows, its rank is 2.  Similarly, the rank of the matrix 
%
\begin{align}
\myvec{1 & 2\\2 & 4} 
\end{align}
is 1, from \ref{eq:line_aug}. 
%
\begin{align}
\because rank \myvec{1 & 2\\2 & 4} \ne rank\myvec{1 & 2 & 4\\2 & 4 & 12},  
\end{align}
%
\eqref{eq:line_mat_eq} has no solution.
%
The equivalent python code is
%
\begin{lstlisting}
codes/line/line_check_sol.py
\end{lstlisting}
%
which plots Fig. \ref{fig:line_check_sol}, which shows that the rails are parallel.
%
\begin{figure}[!ht]
\includegraphics[width=\columnwidth]{./line/figs/line_check_sol.eps}
\caption{}
\label{fig:line_check_sol}
\end{figure}
\item Which of the following pairs of linear equations has a unique solution, no solution, or infinitely many solutions?
%
\begin{enumerate}[itemsep=2pt]
%\begin{multicols}{2}
\item
\begin{align}
\begin{split}
\myvec{1 & -3 }\vec{x}&=3
\\
\myvec{3 & -9 }\vec{x}&=2
\end{split}
\label{linform/11/ab/1.0.1}
\end{align}
\item
\begin{align}
\begin{split}
\myvec{2 & 1 }\vec{x}&=5
\\
\myvec{3 & 2 }\vec{x}&=8
\end{split}
\label{linform/11/ab/1.0.2}
\end{align}
\item
\begin{align}
\begin{split}
\myvec{3 & -5 }\vec{x}&=20
\\
\myvec{6 & -10 }\vec{x}&=40
\end{split}
\end{align}
\item
\begin{align}
\begin{split}
\myvec{1 & -3 }\vec{x}&=7
\\
\myvec{3 & -3 }\vec{x}&=15
\end{split}
\end{align}
%\end{multicols}
\end{enumerate}
%
\solution
\begin{enumerate}
\input{solutions/su2021/2/11/ab/main.tex}    
\input{solutions/su2021/2/11/cd/main.tex}
\end{enumerate}
%
%
\item Find out whether the lines representing the
following pairs of linear equations intersect at a point, are parallel or coincident
%
\begin{enumerate}[itemsep=2pt]
%\begin{multicols}{2}
\item
\begin{align}
\begin{split}
\myvec{5 & -4 }\vec{x}&=-8
\\
\myvec{7 & 6 }\vec{x}&=9
\end{split}
\label{line/6/1.0.1}
\end{align}
\item
\begin{align}
\begin{split}
\myvec{9 & 3 }\vec{x}&=-12
\\
\myvec{18 & 6 }\vec{x}&=-24
\end{split}
\label{line/6/1.0.2}
\end{align}
\item
\begin{align}
\begin{split}
\myvec{6 & -3 }\vec{x}&=-10
\\
\myvec{2 & -1 }\vec{x}&=-9
\end{split}
\label{line/6/1.0.3}
\end{align}
%\end{multicols}
\end{enumerate}
%
\solution
 \input{solutions/su2021/2/6/assignment4.tex}
\item Find the value of $p$ so that the three lines 
%
\begin{align}
\myvec{3 & 1}\vec{x} &= 2
\\
\myvec{p & 2}\vec{x} &= 3
\\
\myvec{2 & -1}\vec{x} &= 3
\end{align}
%
may intersect at one point.
%
\solution
\input{solutions/su2021/2/20/main.tex}
\item Find a condition on $\vec{x}$  such that the points $\vec{x}, \myvec{1\\2}, \myvec{7\\0}$ are collinear.
\\
\solution
%\input{solutions/aug/2/11.tex}
\end{enumerate}
\end{document}


