Forming the augmented matrix,
\begin{align}
	\myvec{
		2 & -3 & 3 & \vrule & 1 & 0 & 0 \\
		2 & 2 & 3 & \vrule & 0 & 1 & 0 \\ 
		3 & -2 & 2 & \vrule & 0 & 0 & 1\\
	}
	\\
	\xleftrightarrow[]{C_2 \leftarrow C_2 + C_1}
	\myvec{
		2 & 0 & 3 & \vrule & 1 & 0 & 0 \\
		2 & 5 & 3 & \vrule & 0 & 1 & 0 \\ 
		3 & 0 & 2 & \vrule & 0 & 1 & 1\\
	}
	\\
	\xleftrightarrow[]{C_1\leftarrow C_3 - C_1}
	\myvec{
		1 & 0 & 3 & \vrule & -1 & 0 & 0 \\
		1 & 5 & 3 & \vrule & 0 & 1 & 0 \\ 
		-1 & 0 & 2 & \vrule & 1 & 1 & 1\\
	}
	\\
	\xleftrightarrow[]{C_3 \leftarrow C_3 - 3C_1}
	\myvec{
		1 & 0 & 0 & \vrule & -1 & 0 & 3 \\
		1 & 5 & 0 & \vrule & 0 & 1 & 0 \\ 
		-1 & 0 & 5 & \vrule & 1 & 1 & -2\\
	}
	\\
	\xleftrightarrow[]{C_3 \leftarrow \frac{1}{5}C_3}
	\myvec{
		1 & 0 & 0 & \vrule & -1 & 0 & 3/5 \\
		0 & 5 & 0 & \vrule & 0 & 1 & 0 \\ 
		-1 & 0 & 1 & \vrule & 1 & 1 & -2/5\\
	}
	\\
	\xleftrightarrow[]{C_2 \leftarrow \frac{1}{5}C_2}
	\myvec{
		1 & 0 & 0 & \vrule & -1 & 0 & 3/5 \\
		0 & 1 & 0 & \vrule & 0 & 1/5 & 0 \\ 
		-1 & 0 & 1 & \vrule & 1 & 1/5 & -2/5\\
	}
\end{align}
yielding
\begin{align}
    \xleftrightarrow[]{C_1 \leftarrow C_1 - C_2}
	\myvec{
		1 & 0 & 0 & \vrule & -1 & 0 & 3/5 \\
		0 & 1 & 0 & \vrule & -1/5 & 1/5 & 0 \\ 
		-1 & 0 & 1 & \vrule & 4/5 & 1/5 & -2/5\\
	}
	\\
	\xleftrightarrow[]{C_1 \leftarrow C_1 + C_3}
	\myvec{
		1 & 0 & 0 & \vrule & -2/5 & 0 & 3/5 \\
		0 & 1 & 0 & \vrule & -1/5 & 1/5 & 0 \\ 
		0 & 0 & 1 & \vrule & 2/5 & 1/5 & -2/5\\
	}
\end{align}
Thus the desired inverse is 
\begin{align}
\vec {A}^{-1} &= {\myvec{-2/5 & 0 & 3/5\\-1/5 & 1/5 & 0\\2/5 & 1/5 & -2/5}}
\end{align}
