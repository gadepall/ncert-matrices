
		From \eqref{eq:gpoly}, 
\begin{equation}
\label{eq:glam}
g\brak{\lambda} = \brak{1+\lambda}^{3}-7\lambda
\end{equation}
%
Also, 
\begin{equation}
 \vec{A}^2 = \vec{A} \implies \vec{A}^2 - \vec{A} = 0 
\end{equation}
Using the Cayley-Hamilton theorem, the eigenvalues satisfy the characteristic equation 
%
\begin{equation}\label{eq:solutions/matrix/77/eq1}
f\brak{\lambda} = \lambda^2 - \lambda = 0 
\end{equation}
$\because  f\brak{\lambda}$  is of degree 2 and $g\brak{\lambda} $ is of degree 3, \eqref{eq:glam} can be expressed as
\begin{align}
	g\brak{\lambda}	 = f\brak{\lambda}		q\brak{\lambda}	+ a\lambda +b
\end{align}
%
where $a, b$ are real numbers and $q\brak{\lambda}$ is some polynomial.  Thus, 
\begin{align}
	g\brak{0}	 &= b = 1
	\\
	g\brak{1}	 &= a+b = 1
	\\
	\implies a &= 90, b = 1
	\label{eq:glam_ab}
\end{align}
Thus, 
\begin{align}
	g\brak{\vec{A}} &= f\brak{\vec{A}}		q\brak{\vec{A}}	+ a\vec{A} +b\vec{I}
	\\
	&= \vec{I}
\end{align}
upon substituting from \eqref{eq:solutions/matrix/77/eq1} and \eqref{eq:glam_ab}.
Option c is the valid answer.

