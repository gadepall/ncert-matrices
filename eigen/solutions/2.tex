If  
 \begin{align}
 \vec A = \myvec{\alpha&\beta\\\gamma&-\alpha},\
  \vec A^2 = I\label{eq:solutions/matrix/75/eq1}
 \end{align}
The characteristic equation is 
\begin{align}
	\vec{|A - \lambda I|} &= 0 
	\\    \implies \mydet{\alpha -\lambda & \beta \\ \gamma & -\alpha -\lambda} &= 0  
	\\\implies (\alpha -\lambda)(-\alpha -\lambda)-\gamma\beta &= 0
	\\\implies \lambda^2-\alpha^2-\gamma\beta &= 0 \label{eq:solutions/matrix/75/eq2}
\end{align}
By the Cayley-Hamilton theorem, every square matrix satisfies its own characteristic equation.
\\Hence,on substituting from \eqref{eq:solutions/matrix/75/eq1} in \eqref{eq:solutions/matrix/75/eq2}
\begin{align}
\vec{A^2-\alpha^2I-\gamma\beta I} &=0.
\\
	\implies \vec{I}\brak{1-\alpha^2-\gamma\beta} &= 0
\end{align}
Hence, (c) is the correct answer


