
Let the balanced version of \eqref{eq:chem_balance} be 
%
\begin{align}
\label{eq:chem_balance_unsol}
x_1Fe+x_2H_2 O &\rightarrow x_3Fe_3 O_4 + x_4H_2
\end{align}
%
which results in the following equations
%
\begin{align}
\begin{split}
\brak{x_1 -3x_3}Fe &= 0
\\
\brak{2x_2 -2x_4}H &= 0
\\
\brak{x_2 -4x_3}O &= 0
\end{split}
\end{align}
which can be expressed as
\begin{align}
\begin{split}
x_1 + 0.x_2 -3x_3 +0.x_4&= 0
\\
0.x_1+2x_2 +0.x_3-2x_4 &= 0
\\
0.x_1+x_2 -4x_3+ 0.x_4 &= 0
\end{split}
\end{align}
%
resulting in the matrix equation
\begin{align}
\label{eq:chem_balance_mat_eq}
\begin{split}
\myvec{
1 & 0 & -3 & 0
\\
0 & 2 & 0 & -2
\\
0 & 1 & -4 & 0
}
\vec{x} &= \vec{0}
\end{split}
\end{align}
%
where
\begin{align}
\vec{x} = \myvec{x_1 \\ x_2 \\ x_3 \\ x_4} 
\end{align}
%\item Solve \eqref{eq:chem_balance_unsol} by row reducing 
%the matrix in \eqref{eq:chem_balance_mat_eq}.
%\\
%\solution  
\eqref{eq:chem_balance_mat_eq} can be row reduced as follows
%
\begin{align}
\label{eq:chem_balance_mat_row}
\myvec{
1 & 0 & -3 & 0
\\
0 & 2 & 0 & -2
\\
0 & 1 & -4 & 0
}
 \xleftrightarrow[]{R_2 \leftarrow \frac{R_2}{2}}
\myvec{
1 & 0 & -3 & 0
\\
0 & 1 & 0 & -1
\\
0 & 1 & -4 & 0
}
\\
 \xleftrightarrow[]{R_3\leftarrow R_3-R_2}
\myvec{
1 & 0 & -3 & 0
\\
0 & 1 & 0 & -1
\\
0 & 0 & -4 & 1
}
\\
 \xleftrightarrow[]{R_1\leftarrow 4R_1-3R_3}
\myvec{
4 & 0 & 0 & -3
\\
0 & 1 & 0 & -1
\\
0 & 0 & -4 & 1
}
\\
 \xleftrightarrow[R_3 \leftarrow -\frac{1}{4}R_3]{R_1\leftarrow \frac{1}{4}}
\myvec{
1 & 0 & 0 & -\frac{3}{4}
\\
0 & 1 & 0 & -1
\\
0 & 0 & 1 & -\frac{1}{4}
}
\end{align}
%
Thus, 
\begin{align}
\label{eq:chem_balance_mat_sol}
x_1 &= \frac{3}{4}x_4, x_2 = x_4, x_3 = \frac{1}{4}x_4
\\
\\
\implies 
\vec{x} &= x_4\myvec{\frac{3}{4} \\ 1 \\ \frac{1}{4} \\ 1}= \myvec{3 \\ 4 \\ 1 \\ 4}
\end{align}
%
upon substituting $x_4 = 4$.
%
\eqref{eq:chem_balance_unsol} then becomes
%
\begin{align}
\label{eq:chem_balance_final}
3Fe+4H_2 O &\rightarrow Fe_3 O_4 + 4H_2
\end{align}

