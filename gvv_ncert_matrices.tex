\let\negmedspace\undefined
\let\negthickspace\undefined
%\RequirePackage{amsmath}
\documentclass[journal,12pt,twocolumn]{IEEEtran}
%
% \usepackage{setspace}
% \usepackage{gensymb}
%\doublespacing
%\singlespacing
%\usepackage{silence}
%Disable all warnings issued by latex starting with "You have..."
%\usepackage{graphicx}
\usepackage{amssymb}
%\usepackage{relsize}
\usepackage[cmex10]{amsmath}
%\usepackage{amsthm}
%\interdisplaylinepenalty=2500
%\savesymbol{iint}
%\usepackage{txfonts}
%\restoresymbol{TXF}{iint}
%\usepackage{wasysym}
\usepackage{amsthm}
%\usepackage{iithtlc}
% \usepackage{mathrsfs}
% \usepackage{txfonts}
% \usepackage{stfloats}
% \usepackage{steinmetz}
% \usepackage{bm}
% \usepackage{cite}
% \usepackage{cases}
% \usepackage{subfig}
%\usepackage{xtab}
\usepackage{longtable}
%\usepackage{multirow}
%\usepackage{algorithm}
%\usepackage{algpseudocode}
\usepackage{enumitem}
 \usepackage{mathtools}
% \usepackage{tikz}
% \usepackage{circuitikz}
% \usepackage{verbatim}
%\usepackage{tfrupee}
\usepackage[breaklinks=true]{hyperref}
%\usepackage{stmaryrd}
%\usepackage{tkz-euclide} % loads  TikZ and tkz-base
%\usetkzobj{all}
\usepackage{listings}
    \usepackage{color}                                            %%
    \usepackage{array}                                            %%
    \usepackage{longtable}                                        %%
    \usepackage{calc}                                             %%
    \usepackage{multirow}                                         %%
    \usepackage{hhline}                                           %%
    \usepackage{ifthen}                                           %%
  %optionally (for landscape tables embedded in another document): %%
    \usepackage{lscape}     
% \usepackage{multicol}
% \usepackage{chngcntr}
%\usepackage{enumerate}

%\usepackage{wasysym}
%\newcounter{MYtempeqncnt}
\DeclareMathOperator*{\Res}{Res}
%\renewcommand{\baselinestretch}{2}
\renewcommand\thesection{\arabic{section}}
\renewcommand\thesubsection{\thesection.\arabic{subsection}}
\renewcommand\thesubsubsection{\thesubsection.\arabic{subsubsection}}

\renewcommand\thesectiondis{\arabic{section}}
\renewcommand\thesubsectiondis{\thesectiondis.\arabic{subsection}}
\renewcommand\thesubsubsectiondis{\thesubsectiondis.\arabic{subsubsection}}

% correct bad hyphenation here
\hyphenation{op-tical net-works semi-conduc-tor}
\def\inputGnumericTable{}                                 %%

\lstset{
%language=C,
frame=single, 
breaklines=true,
columns=fullflexible
}
%\lstset{
%language=tex,
%frame=single, 
%breaklines=true
%}

\begin{document}
%


\newtheorem{theorem}{Theorem}[section]
\newtheorem{problem}{Problem}
\newtheorem{proposition}{Proposition}[section]
\newtheorem{lemma}{Lemma}[section]
\newtheorem{corollary}[theorem]{Corollary}
\newtheorem{example}{Example}[section]
\newtheorem{definition}[problem]{Definition}
%\newtheorem{thm}{Theorem}[section] 
%\newtheorem{defn}[thm]{Definition}
%\newtheorem{algorithm}{Algorithm}[section]
%\newtheorem{cor}{Corollary}
\newcommand{\BEQA}{\begin{eqnarray}}
\newcommand{\EEQA}{\end{eqnarray}}
\newcommand{\define}{\stackrel{\triangle}{=}}

\bibliographystyle{IEEEtran}
%\bibliographystyle{ieeetr}


\providecommand{\mbf}{\mathbf}
\providecommand{\pr}[1]{\ensuremath{\Pr\left(#1\right)}}
\providecommand{\qfunc}[1]{\ensuremath{Q\left(#1\right)}}
\providecommand{\sbrak}[1]{\ensuremath{{}\left[#1\right]}}
\providecommand{\lsbrak}[1]{\ensuremath{{}\left[#1\right.}}
\providecommand{\rsbrak}[1]{\ensuremath{{}\left.#1\right]}}
\providecommand{\brak}[1]{\ensuremath{\left(#1\right)}}
\providecommand{\lbrak}[1]{\ensuremath{\left(#1\right.}}
\providecommand{\rbrak}[1]{\ensuremath{\left.#1\right)}}
\providecommand{\cbrak}[1]{\ensuremath{\left\{#1\right\}}}
\providecommand{\lcbrak}[1]{\ensuremath{\left\{#1\right.}}
\providecommand{\rcbrak}[1]{\ensuremath{\left.#1\right\}}}
\theoremstyle{remark}
\newtheorem{rem}{Remark}
\newcommand{\sgn}{\mathop{\mathrm{sgn}}}
\providecommand{\abs}[1]{\left\vert#1\right\vert}
\providecommand{\res}[1]{\Res\displaylimits_{#1}} 
\providecommand{\norm}[1]{\left\lVert#1\right\rVert}
%\providecommand{\norm}[1]{\lVert#1\rVert}
\providecommand{\mtx}[1]{\mathbf{#1}}
\providecommand{\mean}[1]{E\left[ #1 \right]}
\providecommand{\fourier}{\overset{\mathcal{F}}{ \rightleftharpoons}}
%\providecommand{\hilbert}{\overset{\mathcal{H}}{ \rightleftharpoons}}
\providecommand{\system}{\overset{\mathcal{H}}{ \longleftrightarrow}}
	%\newcommand{\solution}[2]{\textbf{Solution:}{#1}}
\newcommand{\solution}{\noindent \textbf{Solution: }}
\newcommand{\cosec}{\,\text{cosec}\,}
\providecommand{\dec}[2]{\ensuremath{\overset{#1}{\underset{#2}{\gtrless}}}}
\newcommand{\myvec}[1]{\ensuremath{\begin{pmatrix} #1\end{pmatrix}}}
\newcommand{\mydet}[1]{\ensuremath{\begin{vmatrix}#1\end{vmatrix}}}
%\numberwithin{equation}{section}
\numberwithin{equation}{subsection}
%\numberwithin{problem}{section}
%\numberwithin{definition}{section}
\makeatletter
\@addtoreset{figure}{problem}
\makeatother

\let\StandardTheFigure\thefigure
\let\vec\mathbf
%\renewcommand{\thefigure}{\theproblem.\arabic{figure}}
\renewcommand{\thefigure}{\theproblem}
%\setlist[enumerate,1]{before=\renewcommand\theequation{\theenumi.\arabic{align}}
%\counterwithin{align}{enumi}


%\renewcommand{\theequation}{\arabic{subsection}.\arabic{align}}

\def\putbox#1#2#3{\makebox[0in][l]{\makebox[#1][l]{}\raisebox{\baselineskip}[0in][0in]{\raisebox{#2}[0in][0in]{#3}}}}
     \def\rightbox#1{\makebox[0in][r]{#1}}
     \def\centbox#1{\makebox[0in]{#1}}
     \def\topbox#1{\raisebox{-\baselineskip}[0in][0in]{#1}}
     \def\midbox#1{\raisebox{-0.5\baselineskip}[0in][0in]{#1}}

\vspace{3cm}

\title{
	%\logo{
%Computational Approach to School Geometry
Matrices
%	}
}
\author{ G V V Sharma$^{*}$% <-this % stops a space
	\thanks{*The author is with the Department
		of Electrical Engineering, Indian Institute of Technology, Hyderabad
		502285 India e-mail:  gadepall@iith.ac.in. All content in this manual is released under GNU GPL.  Free and open source.}
	
}	
%\title{
%	\logo{Matrix Analysis through Octave}{\begin{center}\includegraphics[scale=.24]{tlc}\end{center}}{}{HAMDSP}
%}


% paper title
% can use linebreaks  \\ [1em]within to get better formatting as desired
%\title{Matrix Analysis through Octave}
%
%
% author names and IEEE memberships
% note positions of commas and nonbreaking spaces ( ~ ) LaTeX will not break
% a structure at a ~ so this keeps an author's name from being broken across
% two lines.
% use \thanks{} to gain access to the first footnote area
% a separate \thanks must be used for each paragraph as LaTeX2e's \thanks
% was not built to handle multiple paragraphs
%

%\author{<-this % stops a space
%\thanks{}}
%}
% note the % following the last \IEEEmembership and also \thanks - 
% these prevent an unwanted space from occurring between the last author name
% and the end of the author line. i.e., if you had this:
% 
% \author{....lastname \thanks{...} \thanks{...} }
%                     ^------------^------------^----Do not want these spaces!
%
% a space would be appended to the last name and could cause every name on that
% line to be shifted left slightly. This is one of those "LaTeX things". For
% instance, "\textbf{A} \textbf{B}" will typeset as "A B" not "AB". To get
% "AB" then you have to do: "\textbf{A}\textbf{B}"
% \thanks is no different in this regard, so shield the last } of each \thanks
% that ends a line with a % and do not let a space in before the next \thanks.
% Spaces after \IEEEmembership other than the last one are OK (and needed) as
% you are supposed to have spaces between the names. For what it is worth,
% this is a minor point as most people would not even notice if the said evil
% space somehow managed to creep in.

%\WarningFilter{latex}{LaTeX Warning: You have requested, on input line 117, version}


% The paper headers
%\markboth{Journal of \LaTeX\ Class Files,~Vol.~6, No.~1, January~2007}%
%{Shell \MakeLowercase{\textit{et al.}}: Bare Demo of IEEEtran.cls for Journals}
% The only time the second header will appear is for the odd numbered pages
% after the title page when using the twoside option.
% 
% *** Note that you probably will NOT want to include the author's ***
% *** name in the headers of peer review papers.                   ***
% You can use \ifCLASSOPTIONpeerreview for conditional compilation here if
% you desire.




% If you want to put a publisher's ID mark on the page you can do it like
% this:
%\IEEEpubid{0000--0000/00\$00.00~\copyright~2007 IEEE}
% Remember, if you use this you must call \IEEEpubidadjcol in the second
% column for its text to clear the IEEEpubid mark.



% make the title area
\maketitle

\newpage

\tableofcontents

\bigskip

\renewcommand{\thefigure}{\theenumi}
\renewcommand{\thetable}{\theenumi}
%\renewcommand{\theequation}{\theenumi}

%\begin{abstract}
%%\boldmath
%In this letter, an algorithm for evaluating the exact analytical bit error rate  (BER)  for the piecewise linear (PL) combiner for  multiple relays is presented. Previous results were available only for upto three relays. The algorithm is unique in the sense that  the actual mathematical expressions, that are prohibitively large, need not be explicitly obtained. The diversity gain due to multiple relays is shown through plots of the analytical BER, well supported by simulations. 
%
%\end{abstract}
% IEEEtran.cls defaults to using nonbold math in the Abstract.
% This preserves the distinction between vectors and scalars. However,
% if the journal you are submitting to favors bold math in the abstract,
% then you can use LaTeX's standard command \boldmath at the very start
% of the abstract to achieve this. Many IEEE journals frown on math
% in the abstract anyway.

% Note that keywords are not normally used for peerreview papers.
%\begin{IEEEkeywords}
%Cooperative diversity, decode and forward, piecewise linear
%\end{IEEEkeywords}



% For peer review papers, you can put extra information on the cover
% page as needed:
% \ifCLASSOPTIONpeerreview
% \begin{center} \bfseries EDICS Category: 3-BBND \end{center}
% \fi
%
% For peerreview papers, this IEEEtran command inserts a page break and
% creates the second title. It will be ignored for other modes.
%\IEEEpeerreviewmaketitle

\begin{abstract}
This book provides a computational approach to school geometry based on the NCERT textbooks from Class 6-12.  Links to sample Python codes are available in the text.  
\end{abstract}
Download python codes using 
\begin{lstlisting}
svn co https://github.com/gadepall/school/trunk/ncert/computation/codes
\end{lstlisting}

%\section{Definitions}
%\subsection{Eigenvalues and Eigenvectors}
%\renewcommand{\theequation}{\theenumi}
%%\begin{enumerate}[label=\arabic*.,ref=\theenumi]
%\begin{enumerate}[label=\thesubsection.\arabic*.,ref=\thesubsection.\theenumi]
%\numberwithin{equation}{enumi}
%\item The eigenvalue $\lambda$ and the eigenvector $\vec{x}$  for a matrix $\vec{A}$ are defined as, 
%\begin{align}
%  \vec{A} \vec{x} = \lambda \vec{x}
%\end{align}
%\item The eigenvalues are calculated by solving the
%equation
%\begin{align}
%  \label{eq:chareq}
%f\brak{\lambda} = \mydet{\lambda \vec{I}- \vec{A} } =0
%\end{align}
%The above equation is known as the characteristic equation.
%\item According to the Cayley-Hamilton theorem,
%\begin{align}
%	\label{eq:cayley}
%  f(\lambda) = 0 \implies f\brak{\vec{A}} = 0
%\end{align}
%\item The trace of a square  matrix is defined to be the sum of the diagonal elements.
%\begin{align}
%	\label{eq:trace}
%	\text{tr}\brak{\vec{A}}=\sum_{i=1}^{N}a_{ii}.
%\end{align}
%	where $a_{ii}$ is the $i$th diagonal element of the matrix $\vec{A}$. 	
%\item The trace of a matrix is equal to the sum of the eigenvalues
%\begin{align}
%	\label{eq:trace_eig}
%	\text{tr}\brak{\vec{A}}=\sum_{i=1}^{N}\lambda_i
%\end{align}
%
%
%\end{enumerate}
%\subsection{Determinants}
%\renewcommand{\theequation}{\theenumi}
%%\begin{enumerate}[label=\arabic*.,ref=\theenumi]
%\begin{enumerate}[label=\thesubsection.\arabic*.,ref=\thesubsection.\theenumi]
%\numberwithin{equation}{enumi}
%
%\item Let 
%\begin{align}
%	\vec{A} = \myvec{a_1 & b_1 & c_1  \\ a_2 & b_2 & c_2  \\ a_3 & b_3 & c_3}.
%\end{align}
%be a $3 \times 3$ matrix. 
%Then, 
%\begin{multline}
%	\mydet{\vec{A}} = a_1 \myvec{ b_2 & c_2 \\  b_3 & c_3} - a_2\myvec{ b_1 & c_1 \\  b_3 & c_3 }  \\ + a_3\myvec{a_1 & b_1 \\ a_2 & b_2 }.
%\end{multline}
%\item Let $\lambda_1,\lambda_2, \dots, \lambda_n$ be the eigenvalues of a matrix $\vec{A}$.  Then,   the product of the eigenvalues is equal to the determinant of $\vec{A}$.
%\begin{align}
%	\mydet{\vec{A}} = \prod_{i=1}^{n}\lambda_i
%\end{align}
%%
%\item 
%\begin{align}
%	\mydet{\vec{A}\vec{B}} = \mydet{\vec{A}}\mydet{\vec{B}}
%\end{align}
%\item If $\vec{A}$ be an $n \times n$ matrix, 
%\begin{align}
%	\mydet{k\vec{A}} = k^n\mydet{\vec{A}}
%\end{align}

%\end{enumerate}
\section{Inverse of a Matrix}
\renewcommand{\theequation}{\theenumi}
%\begin{enumerate}[label=\arabic*.,ref=\theenumi]
\begin{enumerate}[label=\thesection.\arabic*.,ref=\thesection.\theenumi]
\numberwithin{equation}{enumi}
\item For a $2 \times 2$ matrix 
\begin{align}
	\vec{A} = \myvec{a_1 & b_1  \\ a_2 & b_2 },
\end{align}
the inverse is given by 
\begin{align}
	\vec{A}^{-1} = \frac{1}{\mydet{\vec{A}}}\myvec{b_2 & -b_1  \\ -a_2 & a_1 },
\end{align}
\item For higher order matrices, the inverse should be calculated using row operations.
\end{enumerate}
\section{Cayley-Hamilton Theorem}
%\section{Examples}
\renewcommand{\theequation}{\theenumi}
%\begin{enumerate}[label=\arabic*.,ref=\theenumi]
\begin{enumerate}[label=\thesection.\arabic*.,ref=\thesection.\theenumi]
\numberwithin{equation}{enumi}

\item If 
\begin{align}
\vec{A}=\myvec{1 &0 &2\\0 &2 &1\\2 &0 &3},
\end{align}
prove that 
\begin{align}
	\label{eq:chareq1}
	\vec{A}^3-6\vec{A}^2+7\vec{A}+2\vec{I}=0
	\end{align}
	%
\solution 
From   \eqref{eq:chareq}, the characteristic equation is 
\begin{align}
%
 \mydet{1 -\lambda & 0 & 2 \\ 0 & 2 -\lambda & 1 \\ 2 & 0 & 3-\lambda} &= 0  
\end{align}
which can be expanded to obtain 
\begin{align}
%
 (1-\lambda)(2-\lambda)(3-\lambda) + 2(-2(2-\lambda)) &= 0
\end{align}
yielding
\begin{align}
%
 \lambda^3 - 6\lambda^2 + 7\lambda + 2 &= 0
\end{align}
%
upon simplification. 
Using  the Cayley-Hamilton theorem in \eqref{eq:cayley}, 	\eqref{eq:chareq1} is obtained
%
%
\item If 
\begin{align}
	\vec{A}=\myvec{3 &1\\-1 &2},
	\end{align}
	show that 
	\begin{align}
		\label{eq:chareq2}
		\vec{A}^{2}-5\vec{A}+7\vec{I}=0
		\end{align}
	%
\solution
The characteristic equation is 
\begin{align}
\mydet{\vec{A}-\lambda \vec{I}}&=0
\\ \implies \mydet{3-\lambda & 1 \\ -1 & 2-\lambda}&=0
\\ \implies (3-\lambda)(2-\lambda)+1&=0
\\	\text{or, }\lambda^2-5\lambda+7=0
\end{align}
Using the Cayley-Hamilton theorem, \eqref{eq:chareq2} is obtained.
%
\item If $\vec{A}$ is square matrix such that 
\begin{equation}
	\vec{A}^2 = \vec{A},
   \end{equation}
   %
   then 
   \begin{equation}
	\label{eq:gpoly}
	g\brak{\vec{A}} = \brak{\vec{I}+\vec{A}}^{3}-7\vec{A}
   \end{equation}
%
 is equal to
\begin{enumerate}
\item $\vec{A}$ \item  $\vec{I}-\vec{A}$ \item  $\vec{I}$ \item $3\vec{A}$
\end{enumerate}
%
\solution  From \eqref{eq:gpoly}, 
\begin{equation}
\label{eq:glam}
g\brak{\lambda} = \brak{1+\lambda}^{3}-7\lambda
\end{equation}
%
Also, 
\begin{equation}
 \vec{A}^2 = \vec{A} \implies \vec{A}^2 - \vec{A} = 0 
\end{equation}
Using the Cayley-Hamilton theorem, the eigenvalues satisfy the characteristic equation 
%
\begin{equation}\label{eq:solutions/matrix/77/eq1}
f\brak{\lambda} = \lambda^2 - \lambda = 0 
\end{equation}
$\because  f\brak{\lambda}$  is of degree 2 and $g\brak{\lambda} $ is of degree 3, \eqref{eq:glam} can be expressed as
\begin{align}
	g\brak{\lambda}	 = f\brak{\lambda}		q\brak{\lambda}	+ a\lambda +b
\end{align}
%
where $a, b$ are real numbers and $q\brak{\lambda}$ is some polynomial.  Thus, 
\begin{align}
	g\brak{0}	 &= b = 1
	\\
	g\brak{1}	 &= a+b = 1
	\\
	\implies a &= 90, b = 1
	\label{eq:glam_ab}
\end{align}
Thus, 
\begin{align}
	g\brak{\vec{A}} &= f\brak{\vec{A}}		q\brak{\vec{A}}	+ a\vec{A} +b\vec{I}
	\\
	&= \vec{I}
\end{align}
upon substituting from \eqref{eq:solutions/matrix/77/eq1} and \eqref{eq:glam_ab}.
Option c is the valid answer.

\item If 
\begin{align}
	\vec{A}=\myvec{1 &2 &3\\3 &-2 &1\\4 &2 &1}
\end{align}
then show that 
\begin{align}
	\label{eq:chareq4}
	\vec{A}^3-23\vec{A}-40\vec{I}=0
\end{align}
%  
\solution
 The Characteristic equation is given by
\begin{align}
   \abs{\vec{A}-\lambda\vec{I}}&=0
    \\
    \implies \mydet{1-\lambda&2&3\\3&-2-\lambda&1\\4&2&1-\lambda}&=0
    \end{align}
	which can be expressed as 
\begin{multline}
   \implies\brak{1-\lambda}\brak{\brak{-2-\lambda}\brak{1-\lambda}-2}
   \\
   -2\brak{3\brak{1-\lambda}-4}+3\brak{6+4\brak{2+\lambda}}=0
\end{multline}
and simplified to obtain 
\begin{align}
 \implies   \lambda^3-23\lambda-40=0.
\end{align}
%
Using the  Cayley-Hamilton Theorem, 	\eqref{eq:chareq4} is obtained.
%
\end{enumerate}

\section{Matrix Polynomials}
%\section{Examples}
\renewcommand{\theequation}{\theenumi}
%\begin{enumerate}[label=\arabic*.,ref=\theenumi]
\begin{enumerate}[label=\thesection.\arabic*.,ref=\thesection.\theenumi]
\numberwithin{equation}{enumi}

\item Find $\vec{A}^{2}-5\vec{A}+6\vec{I}$,if $\vec{A} = \myvec{2 &0 &1\\2 &1 &3\\1 &-1 &0}$\\
\solution 
Factorizing the corresponding  polynomial, 
\begin{align}
  f(x) &= x^{2}-5x+6 
  \\
  &= \brak{x-3}\brak{x-2}
  \\
  \implies \vec{A}^{2}-5\vec{A}+6\vec{I} &= \brak{\vec{A}-3\vec{I}}(\vec{A}-5\vec{I})
\end{align}
%$\because$
\begin{align}
  \vec{A}-3\vec{I}&=
\myvec{
	2& 0 &1 \\
	2 & 1 &3\\
	1 &-1 & 0
}
+
\myvec{
	-3& 0 &0\\
	0 & -3&0\\
	0 &0 & -3
}
\\
&= \myvec{
	-1 & 0 &1\\
	2 & -2&3\\
	1 &-1 &-3
}
\end{align}
\begin{align}
  \vec{A}-5\vec{I}&=
\myvec{
	2& 0 &1 \\
	2 & 1&3\\
	1 &-1 & 0
}
+
\myvec{
	-5& 0 &0\\
	0& -5 & 0\\
	0 &0 & -5
}
\\
&=\myvec{
	0& 0 &1\\
	2 & -1&3\\
	1 &-1 &-2
}
\end{align}
Multiplying the above expressions, 
\begin{multline}
  \vec{A}^{2}-5\vec{A}+6\vec{I}=
  \\
\myvec{
	-1 & 0 &1\\
	2 & -2&3\\
	1 &-1 &-3
}
\myvec{
	0& 0 &1\\
	2 & -1&3\\
	1 &-1 &-2
}
\\
= \myvec{
	1& -1&-3\\
	-1 & -1&-10\\
	-5 &4 &4
}
\end{multline}
%


\end{enumerate}
\section{Matrix Inverse}
%\subsection{$2\times 1$}
\renewcommand{\theequation}{\theenumi}
%\begin{enumerate}[label=\arabic*.,ref=\theenumi]
\begin{enumerate}[label=\thesection.\arabic*.,ref=\thesection.\theenumi]
\numberwithin{equation}{enumi}
\item Using elementary transforamtions,find the inverse of   
\myvec{1 &-1\\2 &3}\\
  \solution
%\  \input{solutions/su2021/25.tex}
   Given that
    \begin{align}
    \vec{A} = \myvec{1 & -1 \\ 2 & 3}
    \end{align}
    The augmented matrix $ [\vec{A} | \vec{I}]$ is as given below:- 
    \begin{align}
	    \myvec{1 & -1 & \vrule & 1 & 0 \\[1 pt]  2 & 3 & \vrule & 0 & 1}
    \end{align}
    We apply the elementary row operations on $ [\vec{A} | \vec{I}]$ as follows :-
    \begin{align}
    [\vec{A} | \vec{I}] = \myvec{1 & -1 & \vrule & 1 & 0 \\ 2 & 3 & \vrule & 0 & 1}
    \\
    \xleftrightarrow{R_2\leftarrow R_2-2R_1}   
    \myvec{1 & -1 & \vrule & 1 & 0 \\ 0 & 5 & \vrule & -2 & 1}
    \\
    \xleftrightarrow{R_2\leftarrow \frac{R_2}{5}}
    \myvec{1 & -1 & \vrule & 1 & 0 \\ 0 & 1 & \vrule & \frac {-2}{5} & \frac {1}{5}}
    \\
    \xleftrightarrow{R_2\leftarrow R_1+R_2}
    \myvec{1 & 0 & \vrule & \frac{3}{5} & \frac{1}{5}  \\ 0 & 1 & \vrule & \frac{-2}{5} & \frac{1}{5}}
    \end{align}
    By performing elementary transformations on augmented matrix$ [\vec{A} | \vec{I}]$ , we obtained the augmented matrix in the form $ [\vec{I} | \vec{A}]$. 
    Hence we can conclude that the matrix A is invertible and 
    \begin{align}
   \vec{A^{-1}}=\myvec {\frac{3}{5} & \frac{1}{5}  \\  \frac{-2}{5} & \frac{1}{5}} 
    \end{align}
\item Using elementary transforamtions, find the inverse of   
   \myvec{2 &1\\1 &1}\\
  \solution
%  \input{solutions/su2021/26.tex}
Given that
\begin{align}
\vec{A} = \myvec{2 & 1 \\ 1 & 1}
\end{align}
The augmented matrix $ [\vec{A} | \vec{I}]$ is as given below:- 
\begin{align}
\myvec{2 & 1 & \vrule & 1 & 0 \\ 1 & 1 & \vrule & 0 & 1}
\end{align}
We apply the elementary row operations on $ [\vec{A} | \vec{I}]$ as follows :-
\begin{align}
[\vec{A} | \vec{I}] = \myvec{2 & 1 & \vrule & 1 & 0 \\ 1 & 1 & \vrule & 0 & 1}
\\
\xleftrightarrow{R_1\leftarrow R_1-R_2}   
\myvec{1 & 0 & \vrule & 1 & -1 \\ 1 & 1 & \vrule & 0 & 1}
\\
\xleftrightarrow{R_2\leftarrow R_2-R_1}
\myvec{1 & 0 & \vrule & 1 & -1 \\ 0 & 1 & \vrule & -1 & 2}
\end{align}
By performing elementary transformations on augmented matrix$ [\vec{A} | \vec{I}]$ , we obtained the augmented matrix in the form $ [\vec{I} | \vec{A}]$. 
Hence we can conclude that the matrix A is invertible and inverse of the matrix is
\begin{align}
\vec{A^{-1}}=\myvec {1 & -1 \\  -1 & 2} 
\end{align}
\item Obtain the inverse of the following matrix using elementary operations\\
  A=$\myvec{0 &1 &2\\1 &2 &3\\3 &1 &1}$.\\
  \solution
 %   \input{solutions/su2021/69.tex}
    Given that
    \begin{align}
    \vec{A}& =\myvec {0 & 1 & 2\\1 & 2 & 3\\3 & 1 & 1},
    \end{align}
     The augmented matrix $[\vec{A} | \vec{I}]$ is 
    \begin{align}
    \myvec{0 & 1 & 2 & \vrule & 1 & 0 & 0\\1 & 2 & 3 & \vrule & 0 & 1 & 0\\3 & 1 & 1 & \vrule & 0 & 0 & 1}
    \end{align}
    Applying elementary row operations on $[\vec{A} | \vec{I}]$, 
    \begin{align}
    [\vec{A} | \vec{I}] = \myvec{0 & 1 & 2 & \vrule & 1 & 0 & 0\\1 & 2 & 3 & \vrule & 0 & 1 & 0\\3 & 1 & 1 & \vrule & 0 & 0 & 1}
    \\
    \xleftrightarrow{R_1\leftrightarrow R_2}   
    \myvec{1 & 2 & 3 & \vrule & 0 & 1 & 0\\0 & 1 & 2 & \vrule & 1 & 0 & 0\\3 & 1 & 1 & \vrule & 0 & 0 & 1}
    \\
    \xleftrightarrow{R_3\leftarrow R_3-3R_1}   
    \myvec{1 & 2 & 3 & \vrule & 0 & 1 & 0\\0 & 1 & 2 & \vrule & 1 & 0 & 0\\0 & -5 & -8 & \vrule & 0 & -3 & 1}
    \\
    \xleftrightarrow{R_1\leftarrow R_1-2R_2}  
    \myvec{1 & 0 & -1 & \vrule & -2 & 1 & 0\\0 & 1 & 2 & \vrule & 1 & 0 & 0\\0 & -5 & -8 & \vrule & 0 & -3 & 1}
    \\
    \xleftrightarrow{R_3\leftarrow R_3+5R_2}  
    \myvec{1 & 0 & -1 & \vrule & -2 & 1 & 0\\0 & 1 & 2 & \vrule & 1 & 0 & 0\\0 & 0 & 2 & \vrule & 5 & -3 & 1}
    \\
    \xleftrightarrow{R_3\leftarrow R_3/2}
    \myvec{1 & 0 & -1 & \vrule & -2 & 1 & 0\\ 0 & 1 & 2 & \vrule & 1 & 0 & 0\\0 & 0 & 1 &\vrule &\frac{5}{2} &\frac{-3}{2} &\frac{1}{2}}
    \\
    \xleftrightarrow{R_1\leftarrow R_1+R_3}
    \myvec{1 & 0 & 0 & \vrule & \frac{1}{2} & \frac{-1}{2} & \frac{1}{2}\\ 0 & 1 & 2 & \vrule & 1 & 0 & 0\\0 & 0 & 1 &\vrule &\frac{5}{2} &\frac{-3}{2} &\frac{1}{2}}
    \\
    \xleftrightarrow{R_2\leftarrow R_2-2R_3}
    \myvec{1 & 0 & 0 & \vrule & \frac{1}{2} & \frac{-1}{2} & \frac{1}{2}\\ 0 & 1 & 0 & \vrule & -4 & 3 & -1\\0 & 0 & 1 &\vrule &\frac{5}{2} &\frac{-3}{2} &\frac{1}{2}}
    \end{align}
    By performing elementary transformations on augmented matrix$ [\vec{A} | \vec{I}]$, we obtained the augmented matrix in the form $ [\vec{I} | \vec{A}]$. 
    Hence we can conclude that the matrix A is invertible and inverse of the matrix is
    \begin{align}
\vec{A^{-1}}=\myvec { \frac{1}{2} & \frac{-1}{2} & \frac{1}{2} \\  -4 & 3 & -1\\ \frac{5}{2} &\frac{-3}{2} &\frac{1}{2}}
    \end{align}
    \item Find P$^{-1}$, if it exists, given \\
    P=$\myvec{10 &-2\\-5 &1}$.\\
    \solution
Using row reduction, 
%
\begin{align}
\myvec{10 & -2 \\-5 & 1}
\xleftrightarrow {R_2 \leftarrow R_2+\frac{R_1}{2}}\myvec{10& -2& \\0 & 0}
\end{align} 
Since we obtain a zero row, 
$\vec {P}^{-1} $ does not exist.
%    \input{solutions/su2021/70.tex}
 
\end{enumerate}
 
\end{document}

